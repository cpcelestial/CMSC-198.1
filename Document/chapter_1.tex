%   Filename    : chapter_1.tex 
\chapter{Introduction}
\label{sec:researchdesc}    %labels help you reference sections of your document

\section{Overview of the Current State of Technology}
\label{sec:overview}

In the realm of sports science, the enhancement of athletes' response time and agility remains a critical focus, as these attributes directly influence performance in dynamic, unpredictable environments such as team sports and combat disciplines (Hassan et al., 2023). Traditional training methods, including cone drills and ladder exercises, have long been employed to improve these skills, yet they often fall short in replicating the rapid, stimulus-driven demands of real-game scenarios. Over the past decade, light-based reaction training systems—devices utilizing visual stimuli like LED lights to prompt immediate motor responses—have emerged as innovative tools to bridge this gap. Systems such as FITLIGHT, BlazePod, and XLiGHT have been critically analyzed for their design features, including sensor connectivity, battery life, and operational reliability, revealing strengths in portability and customization but limitations in diagnostic precision and validity (Ezhov et al., 2021).

Empirical studies have demonstrated that these systems can significantly enhance visual-motor coordination, reaction speed, and cognitive functions. For instance, interventions using FITLIGHT in small-sided games have led to marked improvements in harmonic abilities (e.g., rhythmization and responsiveness) and basic skills like dribbling among young basketball players (Hassan et al., 2023). Similarly, a 10-week FITLIGHT program improved reaction times and dribbling speeds in female basketball athletes, with effect sizes indicating substantial neural adaptations (Hassan, 2025). In motorsport contexts, light-based reactive agility training has boosted selective attention, cognitive flexibility, and cardiorespiratory capacity in car racing drivers (Horváth et al., 2022). A systematic review of visual training interventions, including light board and stroboscopic methods, further corroborates these benefits, reporting 5-27\% reductions in reaction time across various sports, with greater efficacy in elite and younger athletes (Jothi et al., 2025). Reliability assessments of systems like BlazePod have also affirmed their validity for measuring simple and complex reactions in mixed martial arts (MMA) athletes, with high intraclass correlations supporting their use in training protocols (Polechoński et al., 2024).

Despite these advancements, significant gaps persist in the literature. While light-based systems show promise in controlled settings, their predictive value for field-based reactive agility remains limited, as evidenced by weak correlations between laboratory reaction speeds and on-field performance in soccer players (Broodryk et al., 2025). This suggests a disconnect between isolated visual stimuli and the multifaceted perceptual-cognitive demands of sports, highlighting the need for more integrated, sport-specific designs. Moreover, comparative analyses underscore inconsistencies in system performance, such as variable Bluetooth stability and sensor delays, which could undermine training reproducibility (Ezhov et al., 2021). Long-term efficacy studies are scarce, and few investigations explore the interdisciplinary integrations between sports science and engineering to optimize these technologies.


\section{Problem Statement}
While traditional methods such as cone and ladder drills have been widely used to enhance reaction time and agility, they often fail to simulate the complex, stimulus-driven conditions of real-game scenarios (Hassan et al., 2023). Recent light-based reaction training systems have shown potential in improving visual-motor coordination and cognitive response (Horváth et al., 2022; Jothi et al., 2025). However, these systems remain limited by high costs, connectivity issues, and questionable transferability of laboratory-based improvements to on-field performance (Broodryk et al., 2025; Ezhov et al., 2021).

If these limitations persist, athletes may continue to rely on training tools that inadequately reflect actual gameplay conditions, hindering optimal skill development and competitive performance. Addressing these gaps necessitates the development of a customizable, low-cost, and scientifically validated light-based reaction training system that bridges engineering innovation and applied sports performance research, which our study aims to do as well as evaluate the effectiveness of said system.


\section{Research Objectives}
\label{sec:researchobjectives}

\subsection{General Objective}
\label{sec:generalobjective}

This goal of this study is to develop and evaluate a light-based reaction training system that enhances the response time and agility of athletes, specifically in racket sports such as badminton, tennis, and table tennis.


\subsection{Specific Objectives}
\label{sec:specificobjectives}

\begin{enumerate}
   \item To design and construct a device equipped with infrared sensors, RGB lights, and speakers for accurate motion detection and response measurement, as well as better user experience.
   
   \item To develop a software application that manages the device operations, records performance data, and functions both online and offline.
   
   \item To calibrate and test the system to ensure precision, responsiveness, and synchronization between hardware and software components.
   
   \item To conduct experimental trials assessing the system’s effectiveness in improving athletes’ response time and agility compared to traditional training methods.
   
   \item To evaluate the system’s usability, functionality, and overall user satisfaction based on feedback from athletes and coaches.
\end{enumerate}


\section{Scope and Limitations of the Research}
\label{sec:scopelimitations}

This section discusses the boundaries (with respect to the objectives) of the research and the constraints within 
which the research will be developed.

\begin{comment}

%
% IPR acknowledgement: the sentences inside this comment are from Ethel Ong's slides on Scope and Limitations of the Research
%
Generally, one paragraph should be allotted for each of your research objectives.

Each paragraph contains a brief overview of the concept/theory and the purpose of doing the associated objective.

Each paragraph also includes a description of the scope/limitation of your study.

* Please refer to the slides for examples.

\end{comment}


\section{Significance of the Research}
\label{sec:significance}

This section explains why research must be done in this area.
 It rationalizes the objective of the research with that of the stated problem. 
 Avoid including sentences such as ``This research will be beneficial to the proponent/department/college'' as this is already an inherent requirement of all BSCS majors.  Focus on the research's contribution to the Computer Science field.

The following are guide questions that may help your formulate the significance of your research. 


%
% IPR acknowledgement: the following list of items are from Ethel Ong's slides on Significance of the Research
%
\begin{itemize}
\item  What is the relevance of your work to the computer science community? 

\begin{itemize} 
\item What will be your technical contributions, in terms of algorithms, or approaches, or new domain? 
\item What is your value-added compared to existing systems? 
\end{itemize}

\item What will be your contributions to society in general? 
    \begin{itemize}
      \item Who will benefit from your system? 
      \item Who are your target users and how will this system benefit them? 
   \end{itemize}
\end{itemize}

\begin{comment}
If applicable, describe possible commercialization and/or innovation in your research.
\end{comment}


