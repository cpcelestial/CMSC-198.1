%   Filename    : chapter_3.tex 
\chapter{Research Methodology}
This chapter lists and discusses the specific steps and activities that will be performed  to accomplish the project. 
The discussion covers the activities from pre-proposal to Final SP Writing.

\section{Research Activities}
Research activities include inquiry, survey, research, brainstorming, canvassing, consultation, review, interview, observe, experiment, design, test, document, etc.  
Be sure that for each method, process, or algorithm used, there is a justification why that method was chosen.
The methodology also includes the following information:

\subsection{Research and Consultation}
\begin{itemize}
   \item Resource Persons: UPV PE Department
   \item What: Conduct a comprehensive research on reaction time and agility training, focusing on light- and sound-based systems. Consult experts to validate the feasibility of the proposed system for racket sports
   \item When & Duration: January, 1 week
   \item Where: UPV Covered Court, PE Faculty Room
   \item Why: To establish a strong theoretical foundation, identify current technological gaps, and ensure the design aligns with athletic performance needs and training practices.
\end{itemize}

\subsection{Brainstorming and System Design}
\begin{itemize}
	\item Resource Persons: UPV Computer Science and PE Department
	\item What: Develop a design concept integrating infrared sensors, LED light modules, and speakers for multimodal stimuli, as well as the UI of the software. The brainstorming phase will determine optimal sensor placement and light patterns for accurate response measurement.
	\item When & Duration: January, 1 week
	\item Where: Personal workstations
	\item Why: To conceptualize an efficient, portable, and user-friendly device and software tailored for various racket sports scenarios.
\end{itemize}

\subsection{Prototype Development and Programming}
\begin{itemize}
	\item Resource Persons: Research Adviser
	\item What: Assemble hardware components (IR sensors, microcontroller, LED indicators, and audio modules) and develop an accompanying application for control and data management. Algorithms will be designed to measure response latency and agility metrics accurately.
	\item When & Duration: January-February, 2 weeks
	\item Where: Personal workstations, CAS Computer Labs
	\item Why: To produce a working prototype capable of recording and analyzing athlete response data effectively, validating the proposed system’s technical feasibility.
\end{itemize}


\subsection{Testing and Experimentation}
\begin{itemize}
	\item Resource Persons: UPV PE Department and Athletes
	\item What: Conduct controlled trials to measure response time and agility improvements among participants using the device. Data will be collected across multiple sessions to ensure reliability and repeatability. Statistical analysis will be used to evaluate system effectiveness..
	\item When & Duration: February-March, 3 weekss
	\item Where: UPV Covered Court
	\item Why: To empirically validate the system’s impact on athlete performance and refine device parameters for optimal training outcomes..
\end{itemize}

\subsection{Evaluation and Refinement}
\begin{itemize}
	\item Resource Persons: Technical advisors and sports performance analysts
	\item What: Analyze experimental results, identify potential sources of error, and gather feedback from athletes and coaches. Redesign or recalibrate system components based on user experience and performance data.
	\item When & Duration: March, 2 weeks
	\item Where: Personal workstations, CAS Computer Labs
	\item Why: To improve system accuracy, usability, and durability, ensuring the device meets practical and scientific standards before final deployment.
\end{itemize}

\subsection{Documentation and Reporting}
\begin{itemize}
	\item Resource Persons: Research Adviser
	\item What: Compile all research findings, design specifications, performance data, and analysis into a formal technical report and academic paper.
	\item When & Duration: Entire duration of the study, January-April
	\item Where: Personal workstations
	\item Why: To ensure transparency, replicability, and academic dissemination of the study’s results and methodologies.
\end{itemize}

\section{Calendar of Activities}

A Gantt chart showing the schedule of the activities should be included as a table. For example:

Table \ref{tab:timetableactivities} shows a Gantt chart of the activities.  Each bullet represents approximately
one week worth of activity.

%
%  the following commands will be used for filling up the bullets in the Gantt chart
%
\newcommand{\weekone}{\textbullet}
\newcommand{\weektwo}{\textbullet \textbullet}
\newcommand{\weekthree}{\textbullet \textbullet \textbullet}
\newcommand{\weekfour}{\textbullet \textbullet \textbullet \textbullet}

%
%  alternative to bullet is a star 
%
\begin{comment}
   \newcommand{\weekone}{$\star$}
   \newcommand{\weektwo}{$\star \star$}
   \newcommand{\weekthree}{$\star \star \star$}
   \newcommand{\weekfour}{$\star \star \star \star$ }
\end{comment}



\begin{table}[ht]   %t means place on top, replace with b if you want to place at the bottom
\centering
\caption{Timetable of Activities} \vspace{0.25em}
\begin{tabular}{|p{2in}|c|c|c|c|c|c|c|c|} \hline
\centering Activities (2009) & Jan   & Feb & Mar & Apr & May & Jun & Jul \\ \hline
Study on Prerequisite Knowledge      &   &  & ~~~\weektwo & \weekfour &  &  &  \\ \hline
Review of Existing Racing Strategies & ~~~\weektwo  & \weekfour & \weekfour & \weekfour &  &  &  \\ \hline
Identification of Best Features      &   &  &  & \weekfour & \weektwo~~~ &  &  \\ \hline
Development of Racing Strategies     &   &  &  & ~~~\weektwo & \weekfour & \weektwo~~~ &  \\ \hline
Simulation of Racing Strategies      &   &  &  & ~~~\weektwo & \weekfour & \weekthree~~ &  \\ \hline
Analysis and Interpretation of the Results &   &  &  &  & \weekfour & \weekfour & \weekone~~~~~ \\ \hline
Documentation & ~~~\weektwo  & \weekfour & \weekfour & \weekfour & \weekfour & \weekfour & \weektwo~~~ \\ \hline
\end{tabular}
\label{tab:timetableactivities}
\end{table}

