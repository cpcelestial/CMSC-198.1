%   Filename    : chapter_2.tex 
\chapter{Review of Related Literature}
\label{sec:relatedlit}


\section{Physiological Basis of Reaction and Agility}

Reaction time and agility in athletes are influenced by a complex interplay of neurophysiological factors, including perceptual-cognitive processing, neural pathways, and motor control. Reaction time encompasses the interval from stimulus detection to response initiation, modulated by sensory input (primarily visual), central nervous system processing, and efferent motor commands. Agility, defined as the ability to change direction rapidly while maintaining balance and speed, integrates these with biomechanical elements like strength and coordination (Pojskic et al., 2019). Neurophysiological mechanisms involve the visual cortex for stimulus detection, the prefrontal cortex for decision-making, and the basal ganglia for motor planning, with reaction accuracy linked to efficient attentional orienting and split attention across multiple stimuli (Chow et al., 2022). The prefrontal studies highlight that perceptual factors, such as anticipation and visual search efficiency, contribute significantly, alongside physical attributes like explosive strength and elastic strength (Yıldız et al., 2020). For instance, factor analyses reveal independent components: explosive strength for acceleration, elastic strength for rebound, change-of-direction speed (CODS), and maximal strength. Correlations show that faster reaction times predict superior agility test performances, such as in the Illinois Agility Test or 20-m shuttle sprint, emphasizing the role of neural efficiency (Wang et al., 2024).

Methodologies in these studies often employ systematic reviews and correlational analyses, drawing from diverse athletic populations (e.g., team sports like soccer) (Turna, 2020). Limitations include reliance on lab-based tests that may not fully replicate field conditions, potentially underestimating contextual factors like fatigue or cognitive load (Pojskic et al., 2019). Implications suggest that training targeting neuroplasticity—through repeated stimuli—can enhance synaptic efficiency, reducing reaction latencies and improving agility in sports requiring rapid responses, such as basketball or racing (Chow et al., 2022). Patterns indicate stronger associations in high-level athletes, where cognitive fatigue and sleep quality positively correlate with prolonged reaction times, highlighting the need for holistic training approaches (Yıldız et al., 2020).


\section{Existing Reaction Training Technologies}

Commercial systems like FITLIGHT Trainer and BlazePod represent established light-based technologies for reaction training, with evidence supporting their effectiveness in enhancing athletic performance. FITLIGHT, a wireless LED sensor system, simulates game-like conditions to improve reaction time, reflexes, and cognitive functions (Hassan, 2025). Studies demonstrate its reliability, with test-retest intraclass correlation coefficients (ICC) ranging from 0.81-0.90 and minimal detectable changes in reaction metrics (Steff et al., 2024). In young basketball players, a 10-week FITLIGHT intervention improved executive functions (e.g., inhibition, working memory) and fitness, though gains were comparable to traditional training, suggesting added cognitive demand without superior outcomes (Hassan, 2025). Another trial integrated FITLIGHT into small-sided games for 18 weeks, yielding significant enhancements in coordinative abilities and basic skills, outperforming controls with large effect sizes (Steff et al., 2024).

BlazePod, a pod-based visual-cognitive system, shows similar promise, with excellent reliability in balance activities (Çekok & Anaforoğlu, 2025). A 6-month program in adolescent soccer players improved simple reaction time and cognitive tasks, but between-group differences were non-significant compared to standard training, indicating contextual benefits rather than inherent superiority (Theofilou et al., 2022). Methodologies typically use randomized controlled trials (RCTs) with pre-post assessments, validated via tools like the Stroop Test for cognition and TUG for effort (Çekok & Anaforoğlu, 2025). Limitations include small samples (n = 20-50) and short durations (6-18 weeks), risking overestimation of effects due to novelty; discrepancies arise in massed vs. distributed scheduling, with longer protocols showing sustained gains (Steff et al., 2024b). Implications underscore these systems' role in individualized drills, boosting engagement through gamification, though high costs and proprietary designs limit accessibility for broader athletic populations (Hassan, 2025).


\section{Light- and Sensor-Based Training Research}

Research on light-based stimuli consistently shows enhancements in sports performance, particularly cognitive-motor integration and agility. A 6-week RCT using Witty SEM lights for car racing drivers improved cognitive abilities and cardiorespiratory fitness, with significant group-time interactions (Horváth et al., 2022). In basketball, light stimulation exercises enhanced attention focus (visual/auditory) and skilled hand speed, with pre-post improvements attributed to neuroplastic adaptations (Shimi et al., 2025). VR-adapted light tasks revealed attentional mechanisms, with orienting speed predicting performance, emphasizing split attention for larger stimuli arrays (Shimi et al., 2025). Soccer-specific lighting interventions (6 months) manipulated visual processing, reducing reaction times under varied conditions, though transfer to matches was inferred rather than directly measured (Theofilou et al., 2022).

Methodologies favor RCTs with tools like the Vienna Test System for cognition and breath-by-breath gas analysis for physiology, ensuring objective evidence (Horváth et al., 2022). Limitations encompass single-blinding, uncontrolled nutrition, and lab-based focus, potentially inflating effects; patterns show greater benefits in open-skill sports, with discrepancies in exercise vs. rest conditions (Zhao et al., 2024). Implications highlight light stimuli's potential for dual physical-cognitive gains, transferable to agility-dependent sports, but call for ecological validity in future studies (Shimi et al., 2025).


\section{Identified Gaps in Literature}

The literature reveals notable gaps, particularly the scarcity of open-source, customizable systems for individualized athlete training. While commercial tools like FITLIGHT dominate, they are proprietary and costly, limiting adaptation for diverse needs (Seçkin et al., 2023). Emerging technologies emphasize open-source platforms for tele-exercise, enabling collaborative customization via AI and IoT, yet only 9\% of trials target healthy populations, with inadequate standardization of training parameters (Rebelo et al., 2023). Gaps include high dropout in asynchronous modes, data security concerns, and underexplored adherence factors (Rebelo et al., 2023). Methodologies in gap analyses rely on narrative reviews, highlighting sustainability challenges in open-source projects (Seçkin et al., 2023). This implicates the need for accessible systems to democratize training, addressing disparities in research on healthy athletes.